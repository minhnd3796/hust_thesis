\chapter{First Dummy Chapter}
\section{First Dummy  Section}
The \Gls{latex} typesetting markup language is specially suitable for documents
that include \gls{maths}. \Glspl{formula} are rendered properly an easily once
one gets used to the commands.

\clearpage
\chapter{Second Dummy Chapter}
\section{Second Dummy  Section}
Given a set of numbers, there are elementary methods to compute its
\acrlong{gcd}, which is abbreviated \acrshort{gcd}. This process is similar
to that used for the \acrfull{lcm}.
\begin{equation}\label{eq:Eq1}
    a=b
\end{equation}
\myequations{Equation number \ref{eq:Eq1}}

\clearpage
\chapter{Third Dummy Chapter}
\section{Third Dummy Section}
Using \texttt{biblatex} you can display bibliography divided into sections,
depending of citation type.
Let's cite! Einstein's journal paper \cite{einstein} and the Dirac's book
\cite{dirac} are physics related items.
Next, \textit{The \LaTeX\ Companion} book \cite{latexcompanion}, the Donald 
Knuth's website \cite{knuthwebsite}, \textit{The Comprehensive Tex Archive 
Network} (CTAN) \cite{ctan} are \LaTeX\ related items; but the others Donald 
Knuth's items \cite{knuth-fa,knuth-acp} are dedicated to programming. 
\begin{equation}\label{eq:Eq2}
    b=c
\end{equation}
\myequations{Equation number \ref{eq:Eq2}}