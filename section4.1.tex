\section{Datasets}
\subsection{Vaihingen Dataset}
The data set contains 33 patches (of different sizes), each consisting of a
\acrfull{top_acr} extracted from a larger \acrshort{top_acr} mosaic
(Figure~\ref{fig:vaihingen_tiles}). Numbers refer to the individual patch
numbers as written at the filename endings \cite{vaihingen_isprs}.
\begin{figure}
    \centering
    \includegraphics[width=\textwidth]{vaihingen_tiles}
    \caption{Outlines of all patches given in Table~\ref{tab:vaihingen_2d_table}
    overlaid with the \acrshort{top_acr} mosaic.}
    \label{fig:vaihingen_tiles}
\end{figure}


The ground sampling distance of both, the \acrshort{top_acr} and the
\acrshort{dsm_acr}, is \textbf{9 centimetres}. The \acrshort{dsm_acr} was
generated via dense image matching with Trimble INPHO 5.3 software and Trimble
INPHO OrthoVista was used to generate the \acrshort{top_acr} mosaic. In order to
avoid areas without data (``holes'') in the \acrshort{top_acr} and the
\acrshort{dsm_acr}, the patches were selected from the central part of the
\acrshort{top_acr} mosaic and none at the boundaries. Remaining (very small)
holes in the \acrshort{top_acr} and the \acrshort{dsm_acr} were interpolated.


The \acrshort{top_acr}s are 8-bit \acrfull{tiff} files with three bands; the
three \acrshort{rgb} bands of the \acrshort{tiff} files correspond to the
\emph{near infrared}, \emph{red and green bands} delivered by the camera. The
\acrshort{dsm_acr}s are \acrshort{tiff} files with one band; the grey levels
(corresponding to the \acrshort{dsm_acr} heights) are encoded as 32-bit float
values. The \acrshort{top_acr} and the \acrshort{dsm_acr} are defined on the
same grid, so that it is not necessary to consider the geocoding information in
the processing.


\emph{Labelled \acrfull{gt_acr}} is provided for only one part of the data
(Figure~\ref{fig:examples_top_dsm_gts}). The \acrshort{gt_acr} of the remaining
scenes will remain unreleased and stays with the benchmark test organisers to be
used for evaluation of submitted results. Participants shall use all data with
\acrshort{gt_acr} for training or internal evaluation of their method.
\acrshort{top_acr}: Filename of \gls{top}. \acrshort{dsm_acr}: Filename
of \acrshort{dsm_acr}. $N_{col}$, $N_{row}$: number of columns and rows.
\acrshort{gt_acr}: Filename containing the \gls{gt} (empty if not made
available to participants and used by the benchmark organisers for evaluation
of the results) (Table~\ref{tab:vaihingen_2d_table}).

\begin{table}[h]
    \centering
    \includegraphics[width=\textwidth]{vaihingen_2d_table}
    \caption{Overview of the individual patches (with .tif extensions).}
    \label{tab:vaihingen_2d_table}
\end{table}


Participants are requested to submit labelled images in exactly the same format
as the provided \acrshort{gt_acr} (8bit \acrshort{rgb} tif-files with exactly the
same colours per category) for the part without \acrshort{gt_acr} (patches without
filenames in column GT in the Table). These results will be used for checking
against \acrshort{gt_acr} by the benchmark organisers.

\subsection{Potsdam Dataset}
The data set contains 38 patches (of the same size), each consisting of a
\emph{\acrfull{top_acr}} extracted from a larger \acrshort{top_acr} mosaic
(Figure~\ref{fig:potsdam_tiles}). Numbers refer to the individual patch
numbers as written at the filename endings \cite{potsdam_isprs}.
\begin{figure}
    \centering
    \includegraphics[width=\textwidth]{potsdam_tiles}
    \caption{Outlines of all patches given in Table~\ref{tab:potsdam_2d_table}
    overlaid with the \acrshort{top_acr} mosaic.}
    \label{fig:potsdam_tiles}
\end{figure}


The ground sampling distance of both, the \acrshort{top_acr} and the
\acrshort{dsm_acr}, is \textbf{5 centimetres}. The \acrshort{dsm_acr} was
generated via dense image matching with Trimble INPHO 5.6 software and Trimble
INPHO OrthoVista was used to generate the \acrshort{top_acr} mosaic. In order to
avoid areas without data (``holes'') in the \acrshort{top_acr} and the
\acrshort{dsm_acr}, the patches were selected from the central part of the
\acrshort{top_acr} mosaic and none at the boundaries. Remaining (very small)
holes in the \acrshort{top_acr} and the \acrshort{dsm_acr} were interpolated.
The \acrshort{top_acr} come as \acrshort{tiff} files in different channel
composistions, where each channel has a spectral resolution of 8bit:
\begin{itemize}
    \item IRRG: 3 channels (IR-R-G)
    \item \acrshort{rgb}: 3 channels (R-G-B)
    \item \acrshort{rgb}IR: 4 channels (R-G-B-IR)
\end{itemize}
In this way participants can pick the data needed conveniently.


The \acrshort{dsm_acr}s are \acrshort{tiff} files with one band; the grey levels
(corresponding to the \acrshort{dsm_acr} heights) are encoded as 32-bit float
values. The \acrshort{top_acr} and the \acrshort{dsm_acr} are defined on the
same grid (UTM WGS84) (Figure~\ref{fig:potsdam_top_dsm_label}). Each tile comes
with an affine transformation file (tiff world file) in order to enable a
re-composition of images to larger mosaics if desired.
\begin{figure}
    \centering
    \includegraphics[width=\textwidth]{potsdam_top_dsm_label}
    \caption{Example patches of the semantic object classification contest with
    (a) \gls{top}, (b) \acrshort{dsm_acr}, and (c) ground truth from Potsdam
    dataset \cite{potsdam_isprs}.}
    \label{fig:potsdam_top_dsm_label}
\end{figure}


In addition to the \acrshort{dsm_acr}s so-called \emph{normalised
\acrshort{dsm_acr}s} are provided, that is, after ground filtering the ground
height is removed for each pixel, leading to an representation of heights above
the terrain. This data was produced using some fully automatic filtering
workflow, without manual quality control. Hence, error free data here is not
guaranteed, this is just for researchers to help using height data, other than
the absolute \acrshort{dsm_acr}. In the download folder the corresponding
zip-file will be found. Once unpacked a readme.txt should be read before using
the data. The scripts based on lastools are also contained, so 
might want to tune them. However, support is not available.


\begin{table}[h]
    \centering
    \includegraphics[width=\textwidth]{potsdam_2d_table}
    \caption{Overview of the individual patches (with .tif extensions,
    all images have dimension $6000 \times 6000$ pixels).}
    \label{tab:potsdam_2d_table}
\end{table}

\emph{Labelled \acrfull{gt_acr}} is provided for only one part of the data. The
\acrshort{gt_acr} of the remaining scenes will remain unreleased and stays with
the benchmark test organisers to be used for evaluation of submitted results.
Participants shall use all data with \acrshort{gt_acr} for training or internal
evaluation of their method. Filenames of \acrshort{top_acr} \acrshort{rgb}IR:
\gls{top} with four channels red, green, blue, infrared;
\acrshort{top_acr} IRRG: \gls{top} with three channels infrared, red,
green; \acrshort{top_acr} \acrshort{rgb}: \gls{top} with three channels
red, green, blue; \acrshort{dsm_acr}: \acrfull{dsm_acr}; \acrshort{gt_acr}:
\gls{gt} labels (empty if not made available to participants and used by the
benchmark organisers for evaluation of the results). Note that no gaps exist
between neighbouring image tiles to enable models with context, 
computation of normalised \acrshort{dsm_acr}s etc. The first number always
indicates the row (top to bottom, i.e. north to south) and the second number
the column (left to right, i.e. west to east). For example, direct neighbours
of 5\_12 are 4\_12 (above), 6\_12 (below), 5\_11 (left), and 5\_13 (right). All
images come with corresponding .tfw-files that provide geo-reference in UTM
WGS84 coordinates Table~\ref{tab:potsdam_2d_table}.
