\section{\acrlong{cnn} for Image Classification Problem}
\subsection{Introduction to \acrlong{cnn}}
\textbf{\glspl{cnn}} are very similar to ordinary neural networks have been
seen so far: they are made up of neurons which have learnable weights and
biases. Each neuron receives some inputs, performs a dot product and optionally
follows it with a nonlinearity. The whole network still expresses a single
differentiable score function: from the raw image pixels on one end to class
scores at the other. And they still have a loss function (e.g.
\acrshort{svm}/Softmax) on the last (fully-connected) layer and all the tips
and tricks we developed for learning regular neural networks still apply.


\acrshort{cnn} architectures make the explicit assumption that the inputs are
images, which allows us to encode certain properties into the architecture.
These then make the forward function more efficient to implement and vastly
reduce the amount of parameters in the network.

\subsection{Architecture Overview}
As we saw in previous sections, neural networks receive an input (a single
vector) and transform it through a series of \emph{hidden layers}. Each hidden
layer is made up of a set of neurons, where each neuron is fully connected to
all neurons in the previous layer, and where neurons in a single layer function
completely independently and do not share any connections. The last
fully-connected layer is called the ``output layer'' and in classification
settings it represents the class scores (Figure~\ref{fig:nn_vs_cnn}a).


In CIFAR-10 dataset (Figure~\ref{fig:cifar10}), images are only of
size $[32\times32\times3]$ (32 wide, 32 high, 3 \acrshort{rgb} colour channels),
so a single fully-connected neuron in a first hidden layer of a regular neural
network would have $32\times32\times3=120000$ weights. Moreover, we would
almost certainly want to have serveral such neurons, so the parameters would
add up quickly. Clearly, this full connectivity is wasteful and the huge number
of parameters would quickly lead to overfitting.


\acrlong{cnn} takes advantage of the fact that the input consists of images and
it constrains the architecture in a more sensible way. In particular, unlike a
regular neural Network, the layers of a \acrshort{cnn} have neurons arranged in
3 dimensions: \textbf{width}, \textbf{height}, \textbf{depth}. (Note that the
word \emph{depth} here refers to the third dimension of an activation volume,
not to the depth of a full neural Network, which can refer to the total number
of layers in a network). For example, the input images in CIFAR-10 are an input
volume of activations and the volume has dimensions $[32\times32\times3]$ (width,
height, depth respectively). As we will soon see, the neurons in a layer will
only be connected to a small region of the layer before it, instead of all the
neurons in a fully-connected manner. Moreover, the final output layer would for
CIFAR-10 have dimension $[1\times1\times10]$, because by the end of the
\acrshort{cnn} architecture the full image will be reduced into a single vector
of class scores, arranged along the depth dimension
(Figure~\ref{fig:nn_vs_cnn}b).
\begin{figure}[h]
    \centering
    \begin{subfigure}[b]{0.495\linewidth}
        \includegraphics[width=\linewidth]{neural_net2}
        \caption{A regular 3-layer neural network.}
    \end{subfigure}
    \begin{subfigure}[b]{0.495\linewidth}
        \includegraphics[width=\linewidth]{cnn}
        \caption{
            A \acrshort{cnn} arranges its neurons in three dimensions (width,
            height, depth).
        }
    \end{subfigure}
    \caption{Regular neural network versus \acrlong{cnn}.}
    \label{fig:nn_vs_cnn}
\end{figure}

\subsection{\acrshort{cnn} Layers}
\paragraph*{}
As described above, a simple \acrshort{cnn} is a sequence of layers and every
layer of a \acrshort{cnn} transforms one volume of activations to another
through a differentiable function. Three main types of layers will be used to
build \acrshort{cnn} architectures: \textbf{Convolutional Layer},
\textbf{Pooling Layer} and \textbf{Fully-Connected Layer} (exactly as seen in
regular neural Networks). These layers will be stacked to form a full
\acrshort{cnn} \textbf{architecture}. A simple \acrshort{cnn} for CIFAR-10
classification could have the architecture [INPUT - \acrshort{conv} -
\acrshort{relu} - \acrshort{pool} - \acrshort{fc}]
(Figure~\ref{fig:web_demo_cnn}).
In more detail:
\begin{itemize}
    \item INPUT [32x32x3] will hold the raw pixel values of the image, in this
    case an image of width 32, height 32, and with three colour channels
    \acrshort{rgb}.
    \item \acrshort{conv} layer will compute the output of neurons that are
    connected to local regions in the input, each computing a dot product
    between their weights and a small region they are connected to in the input
    volume. This may result in volume such as $[32\times32\times12]$ if 12
    filters are used.
    \item \acrshort{relu} layer will apply an elementwise activation function,
    such as the $\boldsymbol{max(0, x)}$ thresholding at zero. This leaves the
    size of the volume unchanged ($[32\times32\times12]$).
    \item \acrshort{pool} layer will perform a downsampling operation along the
    spatial dimensions (width, height), resulting in volume such as
    $[16\times16\times12]$.
    \item \acrshort{fc} (i.e. fully-connected) layer will compute the class
    scores, resulting in volume of size $[1\times1\times10]$, where each of the
    10 numbers correspond to a class score, such as among the 10 categories of
    CIFAR-10. As with ordinary neural networks and as the name implies, each
    neuron in this layer will be connected to all the numbers in the previous
    volume.
\end{itemize}


In this way, \acrshort{cnn}s transform the original image layer by layer from
the original pixel values to the final class scores. Note that some layers
contain parameters and other don’t. In particular, the
\acrshort{conv}/\acrshort{fc} layers perform transformations that are a
function of not only the activations in the input volume, but also of the
parameters (the weights and biases of the neurons). On the other hand,
the \acrshort{relu}/\acrshort{pool} layers will implement a fixed function.
The parameters in the \acrshort{conv}/\acrshort{fc} layers will be trained with
gradient descent so that the class scores that the ConvNet computes are
consistent with the labels in the training set for each image.


In summary:
\begin{itemize}
    \item A \acrshort{cnn} architecture is in the simplest case a list of
    layers that transform the image volume into an output volume (e.g. holding
    the class scores).
    \item There are a few distinct types of layers (e.g.
    \acrshort{conv}/\acrshort{fc}/\acrshort{relu}/\acrshort{pool} are by far
    the most popular).
    \item Each layer accepts an input 3D volume and transforms it to an output
    3D volume through a differentiable function.
    \item Each layer may or may not have parameters (e.g.
    \acrshort{conv}/\acrshort{fc} do, \acrshort{relu}/\acrshort{pool} do not).
    \item Each layer may or may not have additional hyperparameters (e.g.
    \acrshort{conv}/\acrshort{fc}/\acrshort{pool} do, \acrshort{relu} does
    not).
\end{itemize}
\begin{figure}[h]
    \centering
    \includegraphics[width=\textwidth]{convnet}
    \caption{An example \acrshort{cnn} architecture.}
    \label{fig:web_demo_cnn}
\end{figure}

\subsubsection{Convolutional Layer}
The convolutional layer is the core building block of a \acrlong{cnn} that does
most of the computational heavy lifting. The convolutional layer's parameters
consists of a set of learnable filters. Every filter is spatially small (along
with width and height), but extends through the full depth of the input volume.
For example, a typical filter on a first layer of a \acrshort{cnn} might have
the size of $[5\times5\times3]$ (i.e. 5 pixels width and height, and 3 pixels
depth because \acrshort{rgb} images have the depth of 3 as the number of colour
channels). During the forward pass, each filter across the width and height of
the input volume is slid (more precisely, convolved) to compute dot products
between the entries of the filter and the input at any position. As sliding the
filter over the width and height of the input volume, a 2-dimensional
activation map will be produced that gives the responses of that filter at
every spatial position. Intuitively, the network will learn the filters that
activate when they see some types of visual feature such an edge or some
orientations or a blotch of some colours on the first layer or eventually
entire honeycomb or wheel-like patterns on higher layers of the network. Now in
an intire set of filters in each convolutional layer (e.g. 12 filters), each of
them will produce a separate 2-dimensional activation map. These activation
maps will be stacked along the depth dimension and produce the output volume.


Every entry in the 3D output volume and also be interpreted as an output of a
biological neuron which looks at a small region in the input and shares
parameters with all neurons to the left and right spatially
(Figure~\ref{fig:cnn_neuron}) (since these numbers all result from applying the
same filter). Each neuron in the convolutional layer is connected only to a
local region in the input volume spatially, but to the full depth (i.e. all
color channels). When dealing with high-dimensional inputs such as images, it is
impractical to connect neurons to all neurons in the previous volume. Instead,
each neuron will be connected to only a local region on the input volume. The
spatial extent of this connectivity is a hyperparameter called
\textbf{receptive field} of the neuron (equivalently this is the filter size).
The extent of the connectivity along the depth axis is always equal to the
depth of the input volume. The connections are local in space (along width
and height) but always full along the entire depth of the input volume.
\begin{figure}[h]
    \centering
    \includegraphics[width=0.5\textwidth]{depthcol}
    \caption{An example input volume in red (e.g. a $[32\times32\times3]$
    CIFAR-10 image), and an example volume of neurons in the first
    convolutional layer.}
    \label{fig:cnn_neuron}
\end{figure}


In terms of spatial arrangement, there are three hyperparameters control the
size of the output volume:
\begin{enumerate}
    \item The \textbf{depth} of the output volume is a hyperparameter: it
    corresponds to the number of filters would be used, each of them when being
    trained will look at something different in the input. For example, if the
    first convolutional layer takes the raw image as input, then different
    neurons along the depth dimension may active in presence of various
    oriented edges, or blobs of colour (Figure~\ref{fig:alex_net_blob}). A set
    of neurons which are all looking the same region of the input will be
    referred as a \textbf{depth column}.
    \item The \textbf{stride} with which the filter is slid must be specified.
    When the stride is 1 then the filters are moved one pixel at a time. When
    the stride is 2 (or uncommonly 3 or more, though this is rare in practice)
    then the filters jump 2 pixels at a time as they are slid around. This will
    produce smaller output volumes spatially.
    \item Sometimes it will be convenient to pad the input volume with zeros
    around the border. The size of this \textbf{zero-padding} is a
    hyperparameter. The nice feature of zero padding is that it will allow us
    to control the spatial size of the output volumes (most commonly as it will
    be used to exactly preserve the spatial size of the input volume so the
    input and output width and height are the same).
\end{enumerate}
\begin{figure}[h]
    \centering
    \includegraphics[width=\textwidth]{cnn2}
    \caption{
        Example filters learned by Krizhevsky et al
        \cite{Krizhevsky:2017:ICD:3098997.3065386}. Each of the 96 filters
        shown here is of size $[11\times11\times3]$, and each one is shared by
        the $55\times55$ neurons in one depth slice.
    }
    \label{fig:alex_net_blob}
\end{figure}


To summarise, the convolutional layer:
\begin{itemize}
    \item Accepts a volume of size $\boldsymbol{W_{1} \times H_{1}
    \times D_{1}}$
    \item Requires four hyperparameters:
    \begin{itemize}
        \item Number of filters $\boldsymbol{K}$,
        \item Their spatial extent $\boldsymbol{F}$,
        \item The stride $\boldsymbol{S}$,
        \item The amount of zero-padding $\boldsymbol{P}$.
    \end{itemize}
    \item Produces a volume of size
    $\boldsymbol{W_{2} \times H_{2} \times D_{2}}$ where:
    \begin{itemize}
        \item $\boldsymbol{W_{2} = (W_{1} - F + 2 \times P) \div S + 1}$
        \item $\boldsymbol{H_{2} = (H_{1} - F + 2 \times P) \div S + 1}$
        (i.e. width and height are computed equally by symmetry)
        \item $\boldsymbol{D_{2} = K}$
    \end{itemize}
    \item With parameter sharing, it introduces
    $\boldsymbol{F \cdot F \cdot D_{1}}$ weights per filter, for a total of
    $\boldsymbol{(F \cdot F \cdot D_{1}) \cdot K}$ weights and $\boldsymbol{K}$
    biases.
    \item In the output volume, the $\boldsymbol{d}$-th depth slice (of size
    $\boldsymbol{W_{2} \times H_{2}}$) is the result of performing a valid
    convolution of the $\boldsymbol{d}$-th filter over the input volume with a
    stride of $\boldsymbol{S}$, and then offset by $\boldsymbol{d}$-th bias.
\end{itemize}
A common setting of the hyperparameters is $\boldsymbol{F = 3}$,
$\boldsymbol{S = 1}$, $\boldsymbol{P = 1}$. However, there are common
conventions and rules of thumb that motivate these hyperparameters.


Below is a running demo of a convolutional layer. Since 3D volumes are hard to
visualise, all the volumes (the input volume (in blue), the weight volumes (in
red), the output volume (in green)) are visualised with each depth slice
stacked in rows. The input volume is of size $\boldsymbol{W_1 = 5}$,
$\boldsymbol{H_1 = 5}$, $\boldsymbol{D_1 = 3}$ and the convolutional layer
parameters are $\boldsymbol{K = 2}$, $\boldsymbol{F = 3}$,
$\boldsymbol{S = 2}$, $\boldsymbol{P = 1}$. That is, we have two filters of
size $3 \times 3$, and they are applied with a stride of 2. Therefore, the
output volume size has spatial size $(5 - 3 + 2) \div 2 + 1 = 3$. Moreover,
notice that a padding of $\boldsymbol{P = 1}$ is applied to the input volume,
making the outer border of the input volume zero. The visualisation below
iterates over the output activations (green), and shows that each element
is computed by elementwise multiplying the highlighted input (blue) with the
filter (red), summing it up, and then offsetting the result by the bias
(Figure~\ref{fig:conv_demo}).
\begin{figure}[h]
    \centering
    \begin{subfigure}[b]{0.325\linewidth}
        \includegraphics[width=\linewidth]{conv_step1}
        \caption{Convolutional step 1.}
    \end{subfigure}
    \begin{subfigure}[b]{0.325\linewidth}
        \includegraphics[width=\linewidth]{conv_step2}
        \caption{Convolutional step 2.}
    \end{subfigure}
    \begin{subfigure}[b]{0.325\linewidth}
        \includegraphics[width=\linewidth]{conv_step3}
        \caption{Convolutional step 3.}
    \end{subfigure}
    \begin{subfigure}[b]{0.325\linewidth}
        \includegraphics[width=\linewidth]{conv_step4}
        \caption{Convolutional step 4.}
    \end{subfigure}
    \begin{subfigure}[b]{0.325\linewidth}
        \includegraphics[width=\linewidth]{conv_step5}
        \caption{Convolutional step 5.}
    \end{subfigure}
    \begin{subfigure}[b]{0.325\linewidth}
        \includegraphics[width=\linewidth]{conv_step6}
        \caption{Convolutional step 6.}
    \end{subfigure}
    \begin{subfigure}[b]{0.325\linewidth}
        \includegraphics[width=\linewidth]{conv_step7}
        \caption{Convolutional step 7.}
    \end{subfigure}
    \begin{subfigure}[b]{0.325\linewidth}
        \includegraphics[width=\linewidth]{conv_step8}
        \caption{Convolutional step 8.}
    \end{subfigure}
    \begin{subfigure}[b]{0.325\linewidth}
        \includegraphics[width=\linewidth]{conv_step9}
        \caption{Convolutional step 9.}
    \end{subfigure}
    \caption{Convolution demo.}
    \label{fig:conv_demo}
\end{figure}

\subsubsection{Pooling Layer}
It is common to periodically insert a pooling layer in-between successive
convolutional layers in a \acrshort{cnn} architecture. Its function is to
progressively reduce the spatial size of the representation to reduce the
amount of parameters and computation in the network, and hence to also control
overfitting. The Pooling Layer operates independently on every depth slice of
the input and resizes it spatially, using the MAX operation. The most common
form is a pooling layer with filters of size $2\times2$ applied with a stride
of 2 downsamples every depth slice in the input by 2 along both width and
height, discarding 75\% of the activations. Every MAX operation would in this
case be taking a max over 4 numbers (little $2\times2$ region in some depth
slice). The depth dimension remains unchanged. More generally, the pooling
layer:
\begin{itemize}
    \item Accepts a volume of size $\boldsymbol{W_1 \times H_1 \times D_1}$
    \item Requires two hyperparameters:
    \begin{itemize}
    \item their spatial extent $F$,
    \item the stride $S$,
    \end{itemize}
    \item Produces a volume of size
    $\boldsymbol{W_2 \times H_2 \times D_2}$ where:
    \begin{itemize}
    \item $\boldsymbol{W_2 = (W_1 - F) \div S + 1}$
    \item $\boldsymbol{H_2 = (H_1 - F) \div S + 1}$
    \item $\boldsymbol{D_2 = D_1}$
    \end{itemize}
    \item Introduces zero parameters since it computes a fixed function of the
    input
    \item Note that it is not common to use zero-padding for pooling layers
\end{itemize}
It is worth noting that there are only two commonly seen variations of the max
pooling layer found in practice: A pooling layer with $\boldsymbol{F=3}$,
$\boldsymbol{S=2}$ (also called overlapping pooling), and more commonly
$\boldsymbol{F=2}$, $\boldsymbol{S=2}$ (Figure~\ref{fig:max_pool_2_2}).
Pooling sizes with larger receptive fields are too destructive.
\begin{figure}[h]
    \centering
    \begin{subfigure}[b]{0.3911299\linewidth}
        \includegraphics[width=\linewidth]{pool}
        \caption{Pooling with filter size 2, stride 2.}
    \end{subfigure}
    \begin{subfigure}[b]{0.5988701\linewidth}
        \includegraphics[width=\linewidth]{maxpool}
        \caption{Max pooling with filter size 2, stride 2.}
    \end{subfigure}
    \caption{Pooling and max pooling.}
    \label{fig:max_pool_2_2}
\end{figure}


\subsubsection{Normalisation Layer}
Many types of normalisation layers have been proposed for use in \acrshort{cnn}
architectures, sometimes with the intentions of implementing inhibition schemes
observed in the biological brain. However, these layers have since fallen out
of favour because in practice their contribution has been shown to be minimal,
if any.

Nevertheless, training deep neural networks is complicated by the fact that the
distribution of each layer's inputs changes during training, as the parameters
of the previous layers change. This slows down the training by requiring lower
learning rates and careful parameter initialisation, and makes it notoriously
hard to train models with saturating nonlinearities. This phenomenon is
referred as internal covariate shift, and address the problem by normalising
layer inputs. This method draws its strength from making normalisation a part
of the model architecture and performing the normalisation for each training
mini-batch (Figure~\ref{fig:batchnorm}). \textbf{Batch Normalisation} allows
us to use much higher learning rates and be less careful about initialisation.
It also acts as a regulariser, in some cases eliminating the need for Dropout
\cite{DBLP:journals/corr/IoffeS15}.
\begin{figure}[h]
    \centering
    \includegraphics[width=0.75\textwidth]{batchnorm}
    \caption{
        Batch Normalising Transform, applied to activation $x$ over a
        mini-batch.
    }
    \label{fig:batchnorm}
\end{figure}


\subsubsection{Fully-Connected Layer}
Neurons in a fully connected layer have full connections to all activations in
the previous layer, as seen in regular neural networks. Their activations can
hence be computed with a matrix multiplication followed by a bias offset.

\subsubsection{Converting Fully-Connected Layers to Convolutional Layers}
It is worth noting that the only difference between \acrshort{fc} and
\acrshort{conv} layers is that the neurons in the \acrshort{conv} layer are
connected only to a local region in the input, and that many of the neurons in
a \acrshort{conv} volume share parameters. However, the neurons in both layers
still compute dot products, so their functional form is identical. Therefore,
it turns out that it is possible to convert between \acrshort{fc} and
\acrshort{conv} layers:
\begin{itemize}
    \item For any \acrshort{conv} layer there is an \acrshort{fc} layer that
    implements the same forward function. The weight matrix would be a large
    matrix that is mostly zero except for at certain blocks (due to local
    connectivity) where the weights in many of the blocks are equal (due to
    parameter sharing).
    \item Conversely, any \acrshort{fc} layer can be converted to a
    \acrshort{conv} layer. For example, an \acrshort{fc} layer with
    $\boldsymbol{K=4096}$ that is looking at some input volume of size
    $\boldsymbol{7\times7\times512}$ can be equivalently expressed as a
    \acrshort{conv} layer with $\boldsymbol{F=7}$, $\boldsymbol{P=0}$,
    $\boldsymbol{S=1}$, $\boldsymbol{K=4096}$. In other words, filter size is
    set to be exactly the size of the input volume, and hence the output will
    simply be $\boldsymbol{1\times1\times4096}$ since only a single depth
    column ``fits'' across the input volume, giving identical result as the
    initial \acrshort{fc} layer.
\end{itemize}



Of these two conversions, the ability to convert an \acrshort{fc} layer to a
\acrshort{conv} layer is particularly useful in practice. Consider a
\acrshort{cnn} architecture that takes a $224\times224\times3$ image, and then
uses a series of \acrshort{conv} layers and \acrshort{pool} layers to reduce
the image to an activations volume of size $7\times7\times512$ (in AlexNet
architecture \cite{Krizhevsky:2017:ICD:3098997.3065386}, this is done by use of
5 pooling layers that downsample the input spatially by a factor of two each
time, making the final spatial size $224 \div 2 \div 2 \div 2 \div 2 \div 2 =
7$). From there, AlexNet uses two \acrshort{fc} layers of size 4096 and
finally the last \acrshort{fc} layers with 1000 neurons that compute the class
scores. Each of these three \acrshort{fc} layers can be converted to
\acrshort{conv} layers as described above:
\begin{itemize}
    \item Replace the first \acrshort{fc} layer that looks at
    $[7\times7\times512]$ volume with a \acrshort{conv} layer that uses filter
    size $\boldsymbol{F=7}$, giving output volume $[1\times1\times4096]$.
    \item Replace the second \acrshort{fc} layer with a \acrshort{conv} layer
    that uses filter size $\boldsymbol{F=1}$, giving output volume
    $[1\times1\times4096]$.
    \item Replace the last \acrshort{fc} layer similarly, with
    $\boldsymbol{F=1}$, giving final output $[1\times1\times1000]$.
\end{itemize}


Each of these conversions could in practice involve manipulating (e.g.
reshaping) the weight matrix $\boldsymbol{W}$ in each \acrshort{fc} layer
into \acrshort{conv} layer filters. It turns out that this conversion allows us
to ``slide'' the original \acrshort{cnn} very efficiently across many spatial
positions in a larger image, in a single forward pass.


Naturally, forwarding the converted \acrshort{cnn} a single time is much more
efficient than iterating the original \acrshort{cnn} over all those 36
locations, since the 36 evaluations share computation. This trick is often
used in practice to get better performance, where for example, it is common to
resize an image to make it bigger, use a converted \acrshort{cnn} to evaluate
the class scores at many spatial positions and then average the class scores.


Lastly, what if we wanted to efficiently apply the original \acrshort{cnn} over
the image but at a stride smaller than 32 pixels? This could be achieved with
multiple forward passes. For example, note that if we wanted to use a stride of
16 pixels we could do so by combining the volumes received by forwarding the
converted \acrshort{cnn} twice: First over the original image and second over
the image but with the image shifted spatially by 16 pixels along both width
and height.

\subsection{\acrshort{cnn} Architectures}
\subsubsection{LeNet}
LeNet-5, a pioneering 7-level convolutional network by LeCun et al in 1998
\cite{Lecun98gradient-basedlearning}, that classifies digits, was applied by
several banks to recognise hand-written numbers on checks (cheques) digitised
in 32x32 pixel greyscale input images (Figure~\ref{fig:lenet5}). The ability
to process higher resolution images requires larger and more convolutional
layers, so this technique is constrained by the availability of computing
resources.
\begin{figure}[h]
    \centering
    \includegraphics[width=\textwidth]{lenet5}
    \caption{LeNet-5 architecture.}
    \label{fig:lenet5}
\end{figure}

\subsubsection{AlexNet}
In 2012, AlexNet \cite{Krizhevsky:2017:ICD:3098997.3065386}
(Figure~\ref{fig:alexnet}) significantly outperformed all the prior competitors
and won the \acrfull{ilsvrc} 2012 \cite{ILSVRC15} (Figure~\ref{fig:ilsvrcs}) by
reducing the top-5 error from 26\% to 15.3\% \cite{ILSVRC12}. The second place
top-5 error rate, which was not a \acrshort{cnn} variation, was around 26.2\%
\cite{ILSVRC12}.
\begin{figure}[h]
    \centering
    \includegraphics[width=\textwidth]{alexnet}
    \caption{AlexNet architecture.}
    \label{fig:alexnet}
\end{figure}

\subsubsection{VGGNet}
The runner-up in \acrshort{ilsvrc} 2014 (Figure~\ref{fig:ilsvrcs}) was the
network from Karen Simonyan and Andrew Zisserman that became known as 
VGGNet \cite{DBLP:journals/corr/SimonyanZ14a}. Its main contribution was in
showing that the depth of the network is a critical component for good
performance. Their final best network contains 16 \acrshort{conv}/\acrshort{fc}
layers and, appealingly, features an extremely homogeneous architecture that
only performs $3\times3$ convolutions and $2\times2$ pooling from the beginning
to the end (Figure~\ref{fig:vgg16}). Their pretrained model is available for
plug and play use in \gls{caffe} \cite{jia2014caffe, caffe_model_zoo}. A
downside of the VGGNet is that it is more expensive to evaluate and uses a lot
more memory and parameters ($140M$). Most of these parameters are in the first
\acrshort{fc} layer, and it was since found that these \acrshort{fc} layers can
be removed with no performance downgrade, significantly reducing the number of
unnecessary parameters.
\begin{figure}[h]
    \centering
    \includegraphics[width=0.75\textwidth]{vgg16}
    \caption{Macro-architecture of VGG16.}
    \label{fig:vgg16}
\end{figure}

\subsubsection{ResNet}
Residual Network developed by Kaiming He et al \cite{DBLP:journals/corr/HeZRS15}
was the winner of \acrshort{ilsvrc} 2015 (Figure~\ref{fig:ilsvrcs}). It features
special skip connections (Figure~\ref{fig:resnets}) and a heavy use of batch
normalisation. The architecture is also missing \acrshort{fc} layers at the end
of the network. The reader is also referred to Kaiming's presentation (video,
slides) and some recent experiments reproduce these networks in Torch. ResNets
are currently by far state-of-the-art \acrfull{cnn} models and are the default
choice for using \acrshort{cnn}s in practice (as of May 10, 2016). In
particular, also see more recent developments that tweak the 
architecture from Kaiming He et al
\cite{DBLP:journals/corr/HeZR016}.
\begin{figure}[h]
    \centering
    \includegraphics[width=\textwidth]{resnets}
    \caption{Residual Network architectures.}
    \label{fig:resnets}
\end{figure}
\begin{figure}[h]
    \centering
    \includegraphics[width=\textwidth]{ilsvrc}
    \caption{Results of winners in the \acrfull{ilsvrc} from 2010 to 2015.}
    \label{fig:ilsvrcs}
\end{figure}
