\section{Introduction to Semantic Image Segmentation Problem}
\subsection{Image Segmentation}
The term \emph{image segmentation} refers to the partition of an image into a
set of regions that cover it (Figure~\ref{fig:football_segmentation}). The goal
in many tasks is for the regions to represent meaningful areas of the image,
such as crops, urban areas and forests of a satellite image. In other analysis
tasks, the regions might be sets of border pixels grouped into such structures
as line segments (Figure~\ref{fig:line_segments}) and circular arc segments
in images of 3D industrial objects. Regions may also be defined as groups of
pixels having both a border and a particular shape such as a circle or ellipse
or polygon. When the interesting regions do not cover the whole image, we can
still talk about segmentation, into foreground regions of interest and
background regions to be ignored \cite{book:28867}.
\begin{figure}[h]
    \centering
    \includegraphics[width=\textwidth]{football_segmentation}
    \caption{
    Football image (left) and segmentation into regions.
    Each region is a set of connected pixels that are similar in colour
    \cite{book:28867}.
    }
    \label{fig:football_segmentation}
\end{figure}


Segmentation has two objectives. The first objective is to decompose the image
into parts for further analysis. In simple cases, the environment might be well
controlled enough so that the segmentation process reliably extracts only the
parts that need to be analysed further. For example, in figure
\ref{fig:face_extraction}, an algorithm was presented for segmenting a human
face from a colour video image. The segmentation is reliable, provided that the
person's clothing or room background does not have the same colour components
as a human face. In complex cases, such as extracting a complete road network
from a grey scale aerial image, the segmentation problem can be very difficult
and might require application of a great deal of domain building knowledge.
The second objective of segmentation is to perform a change of representation.
The pixels of the image must be organised into high-level units that are either
more meaningful or more efficient for further analysis (or both). A critical
issue is whether or not segmentation can be performed for many different
domains using general bottom-up methods that do not use any special domain
knowledge \cite{book:28867}.
\begin{figure}[h]
    \centering
    \includegraphics[width=\textwidth]{line_segments}
    \caption{Blocks image (left) and extracted set of straight line segments
    (right). The line segments were extracted by the ORT (Object Recognition
    Toolkit) package \cite{book:28867}.}
    \label{fig:line_segments}
\end{figure}
\begin{figure}[h]
    \centering
    \includegraphics[width=\textwidth]{face_extraction}
    \caption{
        Face extraction examples: (left) input image, (middle) labelled image,
        (right) boundaries of the extracted face region. (Images from V.
        Bakic \cite{book:28867}.)
    }
    \label{fig:face_extraction}
\end{figure}


\subsection{Semantic Segmentation}
\emph{Semantic Segmentation} is a type of \emph{Image Segmentation} which
describes the process of associating each pixel of an image with a class label,
(such as \textbf{ocean}, \textbf{boat}, \textbf{sky}, or \textbf{land} in
Figure~\ref{fig:four_class_segmentation}) \cite{matlab_segmentation}. More
precisely, \emph{Image Segmentation} is the process of assigning a label to
every pixel in an image such that pixels with the same label share certain
characteristics \cite{tum_segmentation}.


Some of the practical applications of semantic segmentation are:
\begin{itemize}
    \item Object detection \cite{delmerico2011building}
    \begin{itemize}
    \item Face detection
    \item Pedestrian detection
    \item Object location in aerial photography (roads, buildings, vegetation,
    etc.)
    \end{itemize}
    \item Medical imaging \cite{pham2000current}
    \cite{forouzanfar2010parameter}
    \begin{itemize}
    \item Tumour and other pathology location \cite{wu2014brain}
    \cite{george2012mr}
    \item Tissue volume measuring
    \item Diagnosis, study of anatomical structure
        \cite{kamalakannan2010double}
    \item Surgery planning
    \item Virtual surgery simulation
    \end{itemize}
    \item Machine vision
    \item{Traffic control system}
\end{itemize}
Semantic segmentation is a widely used application in many fields, such as Artificial Intelligence and
modern machines.
\begin{figure}[h]
    \centering
    \includegraphics[width=\textwidth]{semanticsegmentation_transferlearning}
    \caption{
    Semantic segmentation of four classes by assigning every picture element
    that belongs to which class \cite{matlab_segmentation}.
    }
    \label{fig:four_class_segmentation}
\end{figure}


An \emph{autonomous car} (Figure~\ref{fig:autonomous_car}) is a vehicle which
is capable of sensing its environment and navigating without human input.
\cite{gehrig1999dead} Autonomous can detect surroundings using a variety of
techniques such as radar, lidar, GPS, odometry and \emph{computer vision}.
Advanced control systems interpret sensory information to identify appropriate
navigation paths, as well as obstacles and relevant signage. Autonomous car
have control systems which are capable of analysing sensory data to distinguish
between different cars on the road, which is critically useful in planning a
path to the desired destination. Semantic segmentation is one of the key
technologies in autonomous car, it provides the fundamental information for
semantic understanding of the video footages. These information are sent to the
advanced control systems which helps the controller to make right decisions
during the trip.
\begin{figure}[h]
    \centering
    \includegraphics[width=\textwidth]{sms}
    \caption{Image segmentation during an autonomous car driving.}
    \label{fig:autonomous_car}
\end{figure}


\emph{Medical imaging} (Figure~\ref{fig:medical_segmentation}) is the technique
and process of creating visual representations of the interior of a body for
clinical analysis and medical intervention, as well as visual representation of
the function of some organs or tissues. Semantic segmentation can extract
clinically useful information from medical images using the power of
\glspl{cnn}. It can be applied to complex medical imaging problems, such as
quantifying damage after traumatic brain injury or organ injury.
\begin{figure}[h]
    \centering
    \includegraphics[width=0.25\textwidth]{medical}
    \caption{Image segmentation for medical imaging.}
    \label{fig:medical_segmentation}
\end{figure}
