\selectlanguage{vietnamese}
% \addcontentsline{toc}{chapter}{Tóm tắt}
\chapter*{Tóm tắt}
Một trong những chủ đề chính của phép quang trắc là cách trích chọn các vật
thể trong dữ liệu khu vực đô thị bằng các cảm biến đường không. Điều làm cho
vấn đề này trở nên khó khăn và thử thách là sự nhập nhằng và nhiễu loạn của
các vật thể như các toà nhà, đường phố, cây cối và xe cộ trong các dữ liệu ảnh
có độ phân giải rất cao. Điều này dẫn đến sự khác nhau tương đối giữa các đối
tượng khác nhau trong cùng một lớp như nhau là cao và giữa các đối tượng khác
nhau trong các lớp khác nhau là thấp. Tập trung vào việc phân vùng theo ý nghĩa
của dữ liệu 2 chiều một cách chi tiết, gán nhãn cho các vật thể thuộc các lớp
khác nhau. Đối tượng nghiên cứu chuyên sâu là dữ liệu với độ phân giải rất cao
từ những cảm biến hình ảnh thế hệ mới và các kĩ thuật xử lý nâng cao phụ thuộc
vào sự phát triển của những thuật toán học máy tiên tiến. Mặc dù vô kể công sức
đã được đầu tư, nhưng các vấn đề trên vẫn có thể được coi là chưa được giải
quyết hoàn toàn. Theo kiến thức cá nhân hiện tại, chưa có một phương pháp hoàn
toàn tự động nào giải quyết bài toán nhận dạng vật thể 2 chiều được ứng dụng
phổ cập rộng rãi mặc dù ít nhất nó đã được cố gắng nghiên cứu và giải quyết ít
nhất trong hai thập kỉ qua.


Có rất nhiều hướng tiếp cận để giải quyết bài toán này, kể cả hướng tiếp cận
học máy hay các thuật toán dự trên luật lệ chính xác. Đồ án này tập trung vào
việc giải quyết vấn đề dưới góc độ một bài toán phân vùng ý nghĩa trên ảnh.
Bài toán này đã được giải quyết trước đó bởi rất nhiều mô hình học máy có giám
sát được đề xuất trong các kết quả nghiên cứu uy tín. Tuy nhiên, các mô hình
đó lại thiếu hụt trong độ chính xác dự đoán và sự chuyển giao mang tính tự động
hóa khi thay đổi các nguồn dữ liệu đầu vào khác nhau. Do đó, đồ án này đã đề
xuất một phương pháp tiếp cận mô hình hóa và giải quyết vấn đề mới mà có thể
chuyển hóa các tri thức cơ bản đã được huấn luyện từ trước trong lĩnh vực thị
giác máy tính nói chung và tinh chỉnh để thích ứng với mọi loại dữ liệu trong
việc giải quyết bài toán phân vùng ảnh theo ý nghĩa.


Trong suốt luận văn đồ án, phương pháp tiếp cận theo một thuật toán học sâu đã
được chọn để sử dụng hai mạng phân loại ảnh điển hình (mạng VGG và ResNet) đã
được áp dụng vào xây dựng các mạng nhân chập hoàn toàn và chuyển giao các thông
tin biểu diễn của dữ liệu qua quá trình học máy bằng việc huấn luyện và tinh
chỉnh lại cho bài toán phân vùng ảnh. Với giải pháp này, không cần tới quá
nhiều kiến thức chuyên môn hẹp đặc biệt để có thể xây dựng được một mô hình với
độ chính xác cao mà chỉ cần đủ dữ liệu tốt. Thực tế, các mô hình được đề xuất
của tôi đã đạt những vị trí khá cạnh tranh trong cuộc thi khoa học Gán nhãn ảnh
Viễn thám 2 chiều cho Hiệp hội Quang trắc và Viễn thám Quốc tế tổ chức.
