\section{Introduction to \Gls{remote-sensing} and \Gls{aerial-photography}}
\subsection{\Gls{remote-sensing}}
The field of \gls{remote-sensing} has been defined many times. Examination of common
elements in these varied definitions permits identification of the topic's most
important themes. From a cursory look at these definitions, it is easy to
identify a central concept: the gathering information at a distance.


Remote sensing is the practice of deriving information about the Earth's land
and water surfaces using images accquired from an overhead perspective, using
electromagnetic radiation in one or more regions of the electromagnetic
spectrum, reflected or emitted from the Earth's surface (Figure
\ref{fig:deathvalley}) \cite{book:731838}.
\begin{figure}[h]
    \centering
    \includegraphics[width=0.25\textwidth]{deathvalley}
    \caption{Death Valley as seen from the Space Shuttle's synthetic aperture
    radar instrument \cite{deathvalley}.}
    \label{fig:deathvalley}
\end{figure}


This definition serves as a concise expression of the scope of this volume.
It is not, however, universally applicable and is not intended to be so,
because practical constraints limit the scope of this volume. So, although
this text must omit many interesting topics (e.g., meteorological or
extraterrestrial \gls{remote-sensing}), it can review knowledge and perspectives
necessary for pursuit of topics that cannot be covered in full here.


Because remotely sensed images are formed by many interrelated precesses,
in isolated focus on any single component will produce a fragmented
understanding. Therefore, our initial view of the field can benefit from
a broad perspective that identifies the kinds of knowledge required for
the practice of \gls{remote-sensing} (Figure~\ref{fig:overview}).
\begin{figure}[h]
    \centering
    \includegraphics[width=0.3\textwidth]{overview}
    \caption{Overview of the \gls{remote-sensing} process.}
    \label{fig:overview}
\end{figure}


Consider first the \emph{physical objects}, consisting of buildings,
vegetation, soil, water and the like. These are the objects that
applications scientists wish to examine. Knowledge of the physical
objects resides within specific disciplines, such as geology, forestry,
soil science, geography and urban planning.


\emph{Sensor data} are formed when an instrument (e.g., a camera or a radar)
views the physical objects by recording electromagnetic radiation emitted or
reflected from the landscape. For many, sensor data often seem to be abstract
and foreign because of their unfamiliar overhead perspective, unusual
resolutions and use of spectral regions outside the visible spectrum. As a
result, effective use of sensor data requires analysis and interpretation to
convert data to information that can be used to address practical problems,
such as sitting landfills or searching for mineral deposits, These
interpretation create \emph{extracted information}, which consists of
transformations of sensor data designed to reveal specific kinds of
information.


Actually, a more realistic view (Figure~\ref{fig:expanded}) illustrates that
the same sensor data can be examined from alternative perspectives to yield
different interpretations. Therefore, a single image can be interpreted to
provide information about soils, land use, or hydrology, for example,
depending on the specific image and the purpose of the analysis. Finally,
we proceed to the \emph{applications}, in which the analysed \gls{remote-sensing}
data can be combined with other data to address a specific practical problem,
such as land-use planning, mineral exploration or water-quality mapping. When
digital \gls{remote-sensing} data are combined with other geospatial data,
applications are implemented in the context of GIS. For example, \gls{remote-sensing}
data may provide accurate land-use information that can be combined with soil,
geologic, transportation and other information to guide the sitting of a new
landfill.
\begin{figure}[h]
    \centering
    \includegraphics[width=\textwidth]{expanded}
    \caption{Expanded view of the process outlined in Figure~\ref{fig:overview}.}
    \label{fig:expanded}
\end{figure}


\subsection{\Gls{aerial-photography}}
Aerial photography is the taking of photographs from an aircraft or other
flying object \cite{aerial_dict}. (Figure
\ref{fig:westerheversand_lighthouse}). The study of \gls{aerial-photography} -
whether it be \gls{photogrammetry} or photo interpretation - is a subset of a
much larger discipline called \emph{\gls{remote-sensing}}. A broad definition of
\gls{remote-sensing} would encompass the use of many different kinds of remote
sensors for the detection of variations in force distributions (compasses
and gravity meters), sound distributions (sonar), microwave distributions
(radar), light distributions (film and digital cameras) and lidar (laser
light).
\begin{figure}[h]
    \centering
    \includegraphics[width=0.75\textwidth]{Leuchtturm_in_Westerheversand_crop}
    \caption{Aerial photograph of Westerheversand Lighthouse.}
    \label{fig:westerheversand_lighthouse}
\end{figure}


\emph{\Gls{photogrammetry}} is the art of science of obtaining reliable quantitative
information (measurements) from aerial photographs \cite{citeulike:4129109}.
\emph{Photo interpretation} is the determination of the nature of objects on
a photograph and the judgment of their significance. For example, the size of
an object is frequently an important consideration in its identification
(Figure~\ref{fig:three_arch_bay}). The end result of photo interpretation is
frequently a thematic map and mapmaking is the primary purpose of
\gls{photogrammetry}. Likewise, \gls{photogrammetry} involves techniques and knowledge of
photo interpretation. For example, the determination of acres of specific
vegetation types requires the interpretation of those types \cite{book:834597}.
\begin{figure}[h]
    \centering
    \includegraphics[width=0.75\textwidth]{three_arch_bay}
    \caption{Three Arch Bay Photo Taken by pilot D Ramey Logan with Mike
    Jarvis.}
    \label{fig:three_arch_bay}
\end{figure}
